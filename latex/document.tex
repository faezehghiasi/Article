\documentclass[11.5pt,onecolumn,a4paper]{article}
\usepackage[hidelinks]{hyperref}
\usepackage{epsfig,graphicx,subfigure,amsthm,amsmath}
\usepackage{color,xcolor}
\usepackage{setspace}
\usepackage{tikz}
\usepackage{xepersian}
\settextfont[Scale=1.2]{BZAR.TTF}
\setlatintextfont[Scale=1]{Times New Roman}
\usepackage[bottom]{footmisc} % Ensure footnotes appear at the bottom of the page

% Custom footnote layout
\makeatletter
\renewcommand\@makefntext[1]{%
	\noindent\hfill#1\makebox[7pt][8]{\, \@thefnmark.}%
}
\makeatother

% Adjust footnote rule and spacing
\renewcommand{\footnoterule}{%
	\kern 5pt % Adjust vertical position of the rule
	\noindent\hfill\rule{0.3\textwidth}{1pt} % Right-aligned horizontal rule
	\kern -1pt % Space below the rule
}
\setlength{\footnotesep}{1em} % Space between footnotes
%\setlength{\skip\footins}{10em} % Space between text and footnotes
\begin{document}
	
	
	\title{بهینه‌سازی خوشه‌بندی و مسیر‌یابی در شبکه‌های حسگر بی‌سیم با مدل‌های \lr{IMD-EACBR}، \lr{EECHS-ISSADE} و \lr{ABC-ACO}} 
	\author{فائزه قیاسی ، رانیا کارگر و ملیکا ملکی\\
		دانشجویان کارشناسی دانشگاه صنعتی اصفهان\\
		مهندسی کامپیوتر}
	\date{}
	\maketitle
	
	\section*{چکیده}
	شبکه‌های حسگر بی‌سیم و اینترنت‌اشیا به دلیل کاربردهای گسترده در حوزه‌هایی همچون نظارت بر محیط، حمل‌ونقل هوشمند و مراقبت‌های بهداشتی، توجه بسیاری را به خود جلب کرده‌اند. یکی از چالش‌های اساسی این شبکه‌ها، محدودیت انرژی گره‌های حسگر و تاثیر آن بر طول عمر شبکه است. در این راستا، استفاده از تکنیک‌های مسیریابی براساس خوشه‌بندی به‌ همراه الگوریتم‌های فراابتکاری برای انتخاب سرخوشه‌ها و طراحی مسیرهای بهینه‌ی انتقال داده‌ها به عنوان راهکاری مناسب مطرح شده‌است. ما در این مقاله سه مدل پیشرفته برای بهینه‌سازی مصرف انرژی و بهبود عملکرد شبکه ارائه می‌دهیم. \lr{IMD-EACBR} که از الگوریتم ارشمیدس بهبودیافته برای انتخاب سرخوشه‌ها و از الگوریتم بهینه‌سازی مبتنی‌بر آموزش و یادگیری اصلاح‌شده‌ی ترکیبی برای مسیریابی چندپرشی استفاده می‌کند، مدل \lr{EECHS-ISSADE} که ترکیبی از الگوریتم‌های جست‌وجوی گنجشک و تکامل تفاضلی را برای خوشه‌بندی مسیریابی پیشنهاد می‌دهد و مدل ترکیبی \lr{ABC-ACO} که از الگوریتم زنبورعسل مصنوعی و کلونی مورچه‌ها برای کاهش تاخیر و توازن مصرف انرژی بهره می‌گیرد. براساس نتایج شبیه‌سازی‌ها مدل‌های پیشنهادی ما در مقایسه با روش‌های پیشین، منجر به افزایش طول عمر شبکه، کاهش مصرف انرژی و بهبود نرخ انتقال داده‌ها می‌شوند. این تحقیق گامی موثر در راستای بهینه‌سازی شبکه‌های حسگر بی‌سیم و اینترنت‌اشیا است.
	\newpage
	\section*{مقدمه}
	\hspace*{1em}در سال‌های اخیر، اینترنت اشیا\lr{\footnote{Internet Of Things}} و شبکه‌های حسگر بی‌سیم\lr{\footnote{Wireless Sensor Networks}} به عنوان دو فناوری نوین و کلیدی، نقش مهمی در زمینه‌های مختلفی از جمله نظارت بر محیط، مراقبت‌های بهداشتی هوشمند، حمل‌ونقل هوشمند و اتوماسیون صنعتی ایفا کرده‌اند \cite{ref1, ref2, ref3}. در این شبکه‌ها، گره‌های حسگر\lr{\footnote{Sensor Nodes}} به‌صورت گسترده در مناطق جغرافیایی پراکنده می‌شوند. این گره‌ها اطلاعات محیطی را جمع‌آوری کرده و به ایستگاه‌ پایه\lr{\footnote{ Base Station}} ارسال می‌کنند. یکی از چالش‌های اصلی این شبکه‌ها، محدودیت انرژی گره‌های حسگر است، چرا که این گره‌ها به طور معمول وابسته به باتری‌های محدود هستند. مصرف سریع انرژی در گره‌ها می‌تواند باعث کاهش طول عمر شبکه\lr{\footnote{Network Lifetime}} و اختلال در انتقال داده‌ها شود. به همین دلیل، بهینه‌سازی مصرف انرژی و طراحی راهکارهایی برای افزایش طول عمر شبکه از مهم‌ترین اولویت‌ها در این حوزه به‌ شمار می‌روند.
	
	\subsection*{تکنیک‌های مسیریابی مبتنی بر خوشه‌بندی}
	\hspace*{1em}یکی از راهکارهای موثر برای مقابله با محدودیت‌های انرژی در شبکه‌های حسگر بی‌سیم، استفاده از تکنیک‌های مسیریابی مبتنی بر خوشه‌بندی است. در این روش، شبکه به چندین خوشه تقسیم می‌شود و برای هر خوشه یک سرخوشه\lr{\footnote{Cluster Head}} تعیین می‌گردد. سرخوشه‌ها وظیفه‌ی جمع‌آوری داده‌ها از گره‌های عضو خوشه و انتقال آن‌ها به ایستگاه پایه را بر عهده دارند. انتخاب بهینه‌ی سرخوشه‌ها و مسیریابی داده‌ها\lr{\footnote{Data Routing}} به منظور کاهش مصرف انرژی، افزایش طول عمر شبکه و بهبود کارایی به‌ویژه در شبکه‌های بزرگ و پیچیده اهمیت زیادی دارند.
	
	\subsection*{الگوریتم‌های بهینه‌سازی برای انتخاب سرخوشه‌ها}
	\hspace*{1em}برای انتخاب بهینه سرخوشه‌ها، الگوریتم‌های فراابتکاری\lr{\footnote{Metaheuristic Algorithms}}متعددی ارائه شده‌اند. این الگوریتم‌ها با استفاده از روش‌های بهینه‌سازی پیشرفته و تحلیل پارامترهای مرتبط، نقش مهمی در کاهش مصرف انرژی و افزایش طول عمر شبکه ایفا می‌کنند. برخی از الگوریتم‌های موثر در این زمینه عبارتند از:
	
	\subsubsection*{\hspace*{1em}\tikz\draw[fill=black,circle] (0,0) circle (3pt); الگوریتم بهینه‌سازی ارشمیدس بهبودیافته\lr{\footnote{Improved Archimedes Optimization Algorithm}} }
	\hspace*{2em}این الگوریتم با بهره‌گیری از یک تابع تناسب\lr{\footnote{Fitness Function  }} که پارامترهایی نظیر فاصله، بهره‌وری انرژی و \hspace*{1em}درجه‌ی گره را در نظر می‌گیرد، سرخوشه‌های بهینه را انتخاب می‌کند.
	
	\subsubsection*{\hspace*{1em}\tikz\draw[fill=black,circle] (0,0) circle (3pt); الگوریتم جست‌وجوی گنجشک\lr{\footnote{Sparrow Search Algorithm }}}
	\hspace*{2em}این الگوریتم از رفتار اجتماعی گنجشک‌ها در یافتن منابع غذایی الهام گرفته و با تحلیل انرژی \hspace*{1em}باقی‌مانده‌ی گره‌ها و فاصله‌ی آن‌ها از ایستگاه پایه، سرخوشه‌های بهینه‌ را تعیین می‌کند.
	
	\subsubsection*{\hspace*{1em}\tikz\draw[fill=black,circle] (0,0) circle (3pt);  الگوریتم تکامل تفاضلی\lr{\footnote{Differential Evolution }}}
	\hspace*{2em}این الگوریتم با بهره‌گیری از روش‌های بهینه‌سازی با پیچیدگی کم و پایدار، فرآیند انتخاب \hspace*{1em}سرخوشه‌ها را تسهیل می‌کند.
	
	\subsubsection*{\hspace*{1em}\tikz\draw[fill=black,circle] (0,0) circle (3pt); الگوریتم زنبور عسل بهبودیافته\lr{\footnote{Improved Artificial Bee Colony }}}
	\hspace*{2em}این الگوریتم با الهام از رفتار زنبورها در یافتن منابع غذایی، گره‌هایی با بیشترین بهره‌وری انرژی \hspace*{1em}و مناسب‌ترین موقعیت مکانی را به عنوان سرخوشه انتخاب می‌کند.
	
	\section*{الگوریتم‌های بهینه‌سازی مسیریابی }
	\hspace*{1em} علاوه بر انتخاب سرخوشه‌ها، طراحی الگوریتم‌های بهینه برای مسیریابی داده‌ها از دیگر چالش‌های اساسی در شبکه‌های حسگر بی‌سیم است. این الگوریتم‌ها با هدف کاهش مصرف انرژی، بهبود تاخیر در انتقال داده‌ها و افزایش کارایی طراحی می‌شوند. برخی از الگوریتم‌های برجسته در این زمینه شامل موارد زیر است: 
	
	\subsubsection*{\hspace*{1em}\tikz\draw[fill=black,circle] (0,0) circle (3pt); الگوریتم مسیریابی چندپرشی مبتنی بر بهینه‌سازی آموزش و یادگیری-اصلاح شده‌ی ترکیبی\lr{\footnote{Teaching-learning-Based Optimization for Multi-Hop Routing  }}}
	\hspace*{2em}این الگوریتم از روش‌های بهینه‌سازی مبتنی برآموزش و یادگیری برای یافتن مسیرهای بهینه، \hspace*{1em}بین گره‌ها استفاده می‌کند.
	
	\subsubsection*{\hspace*{1em}\tikz\draw[fill=black,circle] (0,0) circle (3pt); الگوریتم کلونی مورچه  بهبودیافته\lr{\footnote{Improved Ant Colony Optimization  }}}
	\hspace*{2em}این الگوریتم با ایجاد مسیرهای چندپرشی بهینه، از سرخوشه‌ها به ایستگاه پایه، به کاهش مصرف \hspace*{1em}انرژی کمک می‌کند.
	\newpage
	\subsection*{مدل‌های پیشنهادی}
	\hspace*{1em}در این مقاله، سه مدل پیشنهادی برای بهینه‌سازی خوشه‌بندی و مسیریابی در شبکه‌های حسگر بی‌سیم ارائه می‌دهیم. هر مدل با استفاده از الگوریتم‌های بهینه‌سازی متنوع، سرخوشه‌های بهینه را انتخاب کرده و مسیرهای بهینه برای انتقال داده‌ها را طراحی می‌کند.
	
	
	\subsubsection*{\hspace*{1em}\tikz\draw[fill=black,circle] (0,0) circle (3pt); مدل \lr{IMD-EACBR} \cite{ref4}}
	\hspace*{2em}این مدل از الگوریتم ارشمیدس بهبودیافته برای انتخاب سرخوشه‌ها و از الگوریتم بهینه‌سازی \hspace*{1em}مبتنی بر آموزش و یادگیری اصلاح‌شده برای مسیریابی داده‌ها استفاده می‌کند.هدف این مدل \hspace*{1em}کاهش تاخیر، متوازن‌سازی مصرف انرژی و بهبود کارایی شبکه است.
	
	\subsubsection*{\hspace*{1em}\tikz\draw[fill=black,circle] (0,0) circle (3pt); مدل \lr{EECHS-ISSADE}  \cite{ref5}}
	\hspace*{2em}این مدل از الگوریتم‌های جست‌وجوی گنجشک و تکامل تفاضلی برای انتخاب سرخوشه‌ها و \hspace*{1em}مسیریابی استفاده می‌کند. هدف اصلی این مدل کاهش مصرف انرژی و افزایش طول عمر شبکه \hspace*{1em}از طریق انتخاب بهینه سرخوشه‌ها است.
	
	\subsubsection*{\hspace*{1em}\tikz\draw[fill=black,circle] (0,0) circle (3pt); مدل ترکیبی \lr{ABC-ACO}   \cite{ref6}}
	\hspace*{2em}این مدل از الگوریتم زنبور عسل بهبودیافته برای انتخاب سرخوشه‌ها و از الگوریتم کلونی \hspace*{1em}مورچه‌ها برای مسیریابی بهینه داده‌ها استفاده می‌کند. مکانیزم کنترل درون‌خوشه‌ای نیز برای \hspace*{1em}کاهش مصرف انرژی در گره‌های غیرفعال به این مدل اضافه شده است.

 \hspace*{0em}
  \newline
	\hspace*{1em}این مقاله تلاش دارد تا با طراحی و بررسی مدل‌های پیشنهادی، گام‌های موثری در راستای کاهش مصرف انرژی، افزایش طول عمر شبکه و بهبود عملکرد اینترنت اشیا و شبکه‌های حسگر بی‌سیم بردارد.
	
	\begin{thebibliography}{9}
		\begin{LTR}
			\bibitem{ref1} \lr{Sharma, Deepak, and Amol P. Bhondekar. "Traffic and energy aware routing for heterogeneous wireless sensor networks." \textit{IEEE Communications Letters} 22.8 (2018): 1608-1611.}
			
			\bibitem{ref2} \lr{Farsi, Mohammed, et al. "A congestion-aware clustering and routing (CCR) protocol for mitigating congestion in WSN." \textit{IEEE Access} 7 (2019): 105402-105419.}
			
			\bibitem{ref3} \lr{Satpathy, Sambit, et al. "Design a FPGA, fuzzy based, insolent method for prediction of multi-diseases in rural area." \textit{Journal of Intelligent \& Fuzzy Systems} 37.5 (2019): 7039-7046.}
			
			\bibitem{ref4} \lr{Kathiroli, Panimalar, and Kanmani Selvadurai. "Energy efficient cluster head selection using improved Sparrow Search Algorithm in Wireless Sensor Networks." \textit{Journal of King Saud University-Computer and Information Sciences} 34.10 (2022): 8564-8575.}
			
			\bibitem{ref5} \lr{Lakshmanna, Kuruva, et al. "Improved metaheuristic-driven energy-aware cluster-based routing scheme for IoT-assisted wireless sensor networks." \textit{Sustainability} 14.13 (2022): 7712.}
			
			\bibitem{ref6} \lr{Wang, Zongshan, et al. "An energy efficient routing protocol based on improved artificial bee colony algorithm for wireless sensor networks." \textit{IEEE Access} 8 (2020): 133577-133596.}
		\end{LTR}
	\end{thebibliography}
	
\end{document}
