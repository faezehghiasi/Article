\documentclass{article}
\usepackage[extrafootnotefeatures]{xepersian}

\settextfont{BZAR.TTF}  % Set your desired Persian font
\setlatintextfont{Times New Roman}  % Set your desired Latin font

% Redefine \thefootnote to use Persian numerals in the text
\renewcommand{\thefootnote}{\textpersian{۱}\textpersian{۲}\textpersian{۳}\textpersian{۴}\textpersian{۵}\textpersian{۶}\textpersian{۷}\textpersian{۸}\textpersian{۹}\textpersian{۰}}

% Define a new command for English footnotes
\newcommand{\englishfootnote}[1]{%
	\footnote{#1}%
}

\begin{document}
	
	متن اصلی شما در اینجا قرار می‌گیرد.\footnote{این یک زیرنویس فارسی است.} و این یک متن با شماره پاورقی به زبان فارسی است.\englishfootnote{This is an English footnote.}
	
	متن دیگری با زیرنویس‌های بیشتر.\englishfootnote{Another English footnote.} و این متن شامل شماره‌های فارسی است.
	
\end{document}
